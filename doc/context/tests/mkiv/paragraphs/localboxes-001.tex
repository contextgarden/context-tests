\definelocalboxes
  [linenumber]
  [command=\LeftNumber,
   location=left,
   width=3em,
   style=\bs,
   color=darkred]

\definelocalboxes
  [linenumbertwo]
  [linenumber]
  [command=\RightNumber,
   location=right,
   width=6em,
   style=\bf,
   color=darkgreen]

\definelocalboxes
  [linetext]
  [command=\LeftText,
   location=lefttext,
   style=\bs,
   color=darkblue]

\definelocalboxes
  [linetexttwo]
  [linetext]
  [command=\RightText,
   location=righttext,
   style=\bf,
   color=darkgray]

% just using the content

% \protected\def\LeftNumber {\hbox to \localboxesparameter{width}{\strut\box\localboxcontentbox\hss)}}
% \protected\def\RightNumber{\hbox to \localboxesparameter{width}{\strut(\hss\box\localboxcontentbox)}}

% using the provided line number:

% \def\LineNumberL{\the\localboxlinenumber}
% \def\LineNumberR{\the\localboxlinenumber}

% using a tex counter

% \newcount\MyLineNumberL
% \newcount\MyLineNumberR
% \def\LineNumberL{\global\advance\MyLineNumberL\plusone\the\MyLineNumberL}
% \def\LineNumberR{\global\advance\MyLineNumberR\plusone\the\MyLineNumberR}

% using proper counters

\definecounter[MyLineNumberL]
\definecounter[MyLineNumberR]

\setupcounter[MyLineNumberL][numberconversion=characters]
\setupcounter[MyLineNumberR][numberconversion=romannumerals]

\def\LineNumberL{\incrementcounter[MyLineNumberL]\convertedcounter[MyLineNumberL]}
\def\LineNumberR{\incrementcounter[MyLineNumberR]\convertedcounter[MyLineNumberR]}

\protected\def\LeftNumber {\hbox to \localboxesparameter{width}{\strut(\LineNumberL\hss)}}
\protected\def\RightNumber{\hbox to \localboxesparameter{width}{\strut(\hss\LineNumberR)}}

\protected\def\LeftText #1{#1\quad}
\protected\def\RightText#1{\quad#1}

\showframe

\starttext

\start
    \localbox[linenumber]{}%
    \localbox[linenumbertwo]{}%
    \localbox[linetext]{L}%
    \startlocalbox[linetexttwo]
        R
    \stoplocalbox
    \dorecurse{100}{
        \samplefile{tufte}
        \par
    }
\stop

\stoptext
