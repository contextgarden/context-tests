\continuewhenlmtxmode

% \definebar[mathbackground][backgroundbar][continue=always,max=1,height=\strutht,depth=\strutdp]

\definebar[mybar]  [mathbackground][offset=.250ex,color=red]
\definebar[yourbar][mathbackground][offset=.125ex,color=blue]
\definebar[ourbar] [mathbackground][offset=.125ex,color=green]

\starttext

\startTEXpage[width=4cm]
    \startformula
                a \alignhere= b + c \breakhere
        \mybar {d + \yourbar{e \alignhere=} f_3^2 \breakhere
                g \alignhere \ourbar{=} h} + i
    \stopformula
\stopTEXpage

\samplefile{knuth}
\startformula
    \text{something} + \text{really} + \text{really} + \text{long} \alignhere= \sqrt{x} \numberhere
    \texthere[inbetween]{\input{knuth}}
    d \alignhere= e + f \numberhere \breakhere
    g \alignhere= h + i \breakhere
    j \alignhere= k + l
\stopformula
\samplefile{knuth}

\stoptext
