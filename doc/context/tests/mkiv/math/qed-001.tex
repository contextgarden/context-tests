\continuewhenlmtxmode

\showmakeup[line]

\defineenumeration
  [proof]
  [alternative=serried,
   width=fit,
   distance=1ex,
   text=Proof,
   number=no,
   headstyle=italic,
   headcommand=\groupedcommand{}{.},
   closesymbol=\mathqed]

\starttext

\startproof
Ending with text, the symbols is placed at the end of the line.
\samplefile{knuthmath}
\stopproof

\startproof
A formula in the middle does not change that.
\samplefile{knuthmath}
\startformula
  1 + 1 = 2
\stopformula
\samplefile{knuthmath}
\stopproof

\startproof
The same holds for a numbered formula.
\samplefile{knuthmath}
\startplaceformula
\startformula
  1 + 1 = 2\mtp{.}
\stopformula
\stopplaceformula
\samplefile{knuthmath}
\stopproof

\startproof
If we end with math, there is no symbol placed.
\samplefile{knuthmath}
\startformula
  1 + 1 = 2\mtp{.}
\stopformula
\stopproof

\startproof
We can place it manually with \tex {qedhere}.
\samplefile{knuthmath}
\startformula
  2 + 3 = 5\mtp{.}\qedhere
\stopformula
\stopproof

\page[yes]

\startproof
It also works if we have a formula running over more than one line.
\samplefile{knuthmath}
\startformula
  1 + 2
  \alignhere
  = 2 + 1
  \breakhere
  = 3\mtp{.}
  \qedhere
\stopformula
\stopproof

\startproof
If we have a formula number, we do not want the symbol on the same line.
Then we use \tex {qed} to place the box below.
\samplefile{knuthmath}
\startplaceformula
\startformula
  1 + 1 > 1\mtp{.}\qed
\stopformula
\stopplaceformula
\stopproof

\startproof
That also works for formulas running over more than one line.
\samplefile{knuthmath}
\startplaceformula
\startformula
  1 + 2
  \alignhere
  = 2 + 1
  \breakhere
  = 3\mtp{.}\qed
\stopformula
\stopplaceformula
\stopproof

\startproof
It also works for alignments.
\samplefile{knuthmath}
\startplaceformula
\startformula
  \startalign
    \NC f(x) \NC = \sin x \NR[+]
    \NC g(x) \NC = \cos x \NR[+]
  \stopalign
  \qed
\stopformula
\stopplaceformula
\stopproof

\startproof
We can also force the symbol on an earlier line.
\samplefile{knuthmath}
\startformula
  1 + 2 = 2 + 1\qedhere
  \breakhere
  = 3 - 0
\stopformula
\stopproof

\page[yes]

\startproof
Just another example with the \tex {numberhere} mechanism.
\samplefile{knuthmath}
\startformula
  1 + 2 \alignhere
  = 2 + 1 \numberhere \breakhere
  = 3\mtp{.} \numberhere \qed
\stopformula
\stopproof

\startproof
And a final example.
\samplefile{knuthmath}
\startformula
  1 + 2 \alignhere
  = 2 + 1 \numberhere[eq:foo] \breakhere
  = 3\mtp{.} \qedhere
\stopformula
\stopproof

See \in[eq:foo]
\stoptext
