\defineblock[openquestion]
\defineblock[itemintro]
\defineblock[question]
\defineblock[answer]
\defineblock[explanation]

\setupblock
  [openquestion]
  [before={\directsetup{QuestionNumber}}]

\definelabel
  [Exercise] % or enumeration

\startsetups QuestionNumber
    \Exercise
\stopsetups

\usemodule[dejavu,art-01]

\starttext

% \hideblocks[openquestion]
\keepblocks[openquestion,itemintro,question,answer,explanation]
%  \useblocks[openquestion,itemintro]
% \selectblocks[itemintro,question]

\subject{Questions}

\hideblocks[answer,explanation]

\startsetups tex:openquestion
    \keepblocks[answer]
\stopsetups

\startsetups tex:explanation
    \hideblocks[answer]
\stopsetups

\setupblock[openquestion][inner=\directsetup{tex:openquestion}]
\setupblock[explanation] [inner=\directsetup{tex:explanation}]

\beginopenquestion

  \beginitemintro
    This is the introduction to the question.
  \enditemintro

  \beginquestion
    This is the question itself. Isn't it?
  \endquestion

  \beginexplanation
    This is the explanation.
  \endexplanation

  \beginanswer
    This is the answer.
  \endanswer

\endopenquestion

\beginopenquestion

  \beginitemintro
    This is the introduction to the question.
  \enditemintro

  \beginquestion
    This is the question itself. Isn't it?
  \endquestion

  \beginexplanation
    This is the explanation.
  \endexplanation

\endopenquestion

\beginopenquestion

  \beginitemintro
    This is the introduction to the question.
  \enditemintro

  \beginquestion
    This is the question itself. Isn't it?
  \endquestion

  \beginanswer
    This is the answer.
  \endanswer

\endopenquestion

\subject{Answers}

\resetcounter[Exercise]

\keepblocks[openquestion,answer,explanation]
\hideblocks[itemintro,question]
\useblocks[openquestion]

\blank[2*big]

\textrule{Mijn vraag}

\blank

Indien er een 'explanation' is: dan explanation dus geen answer.

\blank

Een soort conditional useblock.

\blank

Dus: bij vraag 1 wil ik graag alleen de uitleg vergelijk Math4All.

\stoptext
