% example by thomas

\starttext

\setupTABLE [row]    [each] [align={lohi,middle}]
\setupTABLE [column] [each] [width=2cm]
\setupTABLE [column] [1]    [width=5cm]

\bTABLE
\bTR[height=0pt,offset=overlay,frame=off] \dorecurse{7}{\bTD\eTD} \eTR
\bTR \bTD OSI-Schicht            \eTD \bTD[nc=6] Umsetzung    \eTD \eTR
\bTR \bTD Anwendungsschicht      \eTD \bTD[nc=2,nr=3] SOME/I  \eTD
                                      \bTD[nc=2,nr=3] AVB/TSN \eTD
                                      \bTD[nc=2,nr=3] DoIP    \eTD \eTR
\bTR \bTD Darstellungsschicht    \eTD \eTR
\bTR \bTD Sitzungsschicht        \eTD \eTR
\bTR \bTD Transportschicht       \eTD \bTD[nc=3] TCP        \eTD
                                      \bTD[nc=3] UDP        \eTD \eTR
\bTR \bTD Vermittlungsschicht    \eTD \bTD[nc=3] IPv4       \eTD
                                      \bTD[nc=3] IPv6       \eTD \eTR
\bTR \bTD Sicherungsschicht      \eTD \bTD[nc=6] Ethernet   \eTD \eTR
\bTR \bTD Bitübertragungsschicht \eTD \bTD[nc=2] 100BASE-TX \eTD
                                      \bTD[nc=2] 100BASE-T  \eTD
                                      \bTD[nc=2] 1000BASE-T \eTD \eTR
\eTABLE

\stoptext
