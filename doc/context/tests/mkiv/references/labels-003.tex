\setupinteraction
  [state=start,
   focus=standard]

\definereferenceformat
  [myref]
  [label=reference:*]

\setuplabeltext
  [en]
  [theorem=Theorem~,
   lemma=Lemma~]

\setuplabeltext
  [en]
  [reference:section=RefSection~,
   reference:formula=RefFormula~,
  %reference:lemma=RefLemma~,
   reference:theorem=RefTheorem~,
  ]

% \setuplabeltext
%   [en]
%   [reference:section=Section~,
%    reference:formula=Formula~]

\setuphead
  [section]
  [bodypartlabel=]

\setupcounter
  [formula]
  [label=]

\defineformula
  [nice]

\defineenumeration
  [theorem]
  [text=, % Theorem,
   label=theorem,
   alternative=serried,
   width=fit,
   way=bysection,
   prefix=yes,
   prefixsegments=section]

\defineenumeration
  [lemma]
  [theorem]
  [%text=Lemma,
   label=lemma,
  ]

% \defineenumeration
%   [lemma]
%   [text=Lemma,
%    label=lemma,
%    counter=theorem, %yes or no
%    alternative=serried,
%    width=fit,
%    way=bysection,
%    prefix=yes,
%    prefixsegments=section]

\starttext

\startsection
  [title=Some title,
   reference=section:some]

See \myref[section:some].

\startformula
  1 + 1 = 2 \numberhere[eq:one]
\stopformula

See \myref[eq:one].

\startniceformula
  3 + 3 = 6 \numberhere[eq:two]
\stopniceformula

See \myref[eq:two].

\starttheorem[reference=lem:testtheorem]
    Not all \m {y} eat \m {x}.
\stoptheorem

\startlemma[reference=lem:testlemma]
    But all \m {x} eat \m {y}.
\stoplemma

As we see in \myref[lem:testlemma], the situation is fine.

\blank lemmas:\blank

\placelist[lemma][criterium=text]

\blank theorems:\blank

\placelist[theorem][criterium=text]

\blank both:\blank

\placelist[theorem,lemma][criterium=text]

\stopsection

\stoptext
