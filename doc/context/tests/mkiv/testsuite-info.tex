% language=us

\usemodule[article-basic,abbreviations-logos]

\noheaderandfooterlines

\starttext

\startsubject[title=The test suite]

This suite is a loosely organized collection of small files. Some files are tests
that we made during development, some are examples taken (and/or adapted) from
the mailing list and mail exchanges. Many of the files show some tricks that are
not really meant for daily usage.

If you use these files as a kind of regression test, an exact comparison with
previous runs not always makes sense. Some files will differ due to typeset
times, or because random content is used. Also some functionality can have been
improved or fonts be updated. The \type {testsuite-info.lua} file provides some
information about what files to skip in such a test.

I run this suite occasionally when we have done some fundamental update to
\CONTEXT\ or \LUATEX. When I'm in a hurry the test is simply running all files
and checking if we get an error. Sometimes I do a more extensive test with the
\type {mtx-testsuite} script that can be run as follows:

\starttyping
mtxrun --script testsuite --compare --objects \
    --oldname=cld-compare-old  --newname=cld-compare-new  --pattern=**/*.cld
mtxrun --script testsuite --compare --objects \
    --oldname=mkiv-compare-old --newname=mkiv-compare-new --pattern=**/*.mkiv
mtxrun --script testsuite --compare --objects \
    --oldname=mkvi-compare-old --newname=mkvi-compare-new --pattern=**/*.mkvi
mtxrun --script testsuite --compare --objects \
    --oldname=tex-compare-old  --newname=tex-compare-new  --pattern=**/*.tex
\stoptyping

This is just one way of comparing results and probably not the best one either
but it evolved that way. If you have your own test method, it might help to know
that you can run \CONTEXT\ in a way that gives (in most cases) comparable \PDF\
files:

\starttyping
context --batch --nocompression --nodates --trailerid=1 somefile
\stoptyping

You can also run \type {contextjit} but that might fail on extremely large fonts
due to limitations in the virtual machine.

Several users run these files as test and report me issues. Feel free to send
reports or addenda for the info file. If you observe differences, first make sure
that the newer results are okay: after all macros might have been improved. We
assume that \CONTEXT\ \LMTX\ with \LUAMETATEX\ is used, but \LUATEX\ should work
fine for most files.

\blank[2*big]

\startlines
Hans Hagen
Hasselt NL
\stoplines

\stopsubject

\stoptext
