\starttext

\setuplinenumbering[start=20,step=5,continue=no,method=next]

\startlinenumbering
Thus, I came to the conclusion that the designer of a new system
must not only be the implementer and first large--scale
user;\someline [here] the designer should also write the first
user manual.

The separation of any of these four components would have hurt
\TeX\ significantly. If I had not participated fully in all these
activities, literally hundreds of improvements would never have
\someline[there]been made, because I would never have thought of
them or perceived why they were important.

But a system \startline[range]cannot be successful if it is too
strongly influenced by a single person. Once the initial design is
complete and fairly robust, the real test begins as people with
many different\stopline [range] viewpoints undertake their own
experiments. (DEK)
\stoplinenumbering

\blank oeps \blank

\startlinenumbering
Thus, I came to the conclusion that the designer of a new system
must not only be the implementer and first large--scale
user; the designer should also write the first
user manual.

The separation of any of these four components would have hurt
\TeX\ significantly. If I had not participated fully in all these
activities, literally hundreds of improvements would never have
been made, because I would never have thought of
them or perceived why they were important.

But a system cannot be successful if it is too
strongly influenced by a single person. Once the initial design is
complete and fairly robust, the real test begins as people with
many different viewpoints undertake their own
experiments. (DEK)
\stoplinenumbering

\blank

see at \inline{line}[here], \inline{line}[there] and also at
\inline [range]

\stoptext



Issues:

I. Non-bidi issues

0. very first linenote range is incorrect; says 100--109, not correct. seems
to add up the other environments.

1. Default formatting is offset from the margin, but you know that already :-)

2. Can we increment line numerals? eg

\setuplinenumbering[increment=5,next=yes]

Increment by 5 and ignore the first line numeral:

	blah blah
	blah blah
	blah blah
	blah blah
5	blah blah
	blah blah
	blah blah
	blah blah
	blah blah
10	blah blah

3. Can we change the size or style of the main line numerals? eg, oldstyle or a different font?

4. Can we format the notes?

a. multicolumn notes with n columns -- eg p. 396 of the TeXBook

b. paragraphed notes -- eg p. 398

c. Is the first argument [<reference>] for \start-\stoplinenote mandatory?
For lots and lots of notes it can be unwieldy.

d. Can we add a note seperator line between note classes, eg linenote and extralinenote layers? Can we control the internote glue?

e. Can we control the formtting of the note seperator line? Choose another
seperator altogether, like a figure?

f. Can we add a(n optional) lemma argument, eg

\setuplinenote[lemmasep={]}]

\startlinenote [one][A linenote\textellipsis 3--5] {The lemma is followed by the actual note.}
A linenote environment has a line range. That is, if the environment is, eg,
3 lines long, then we'll get a number range in the margin such as 3--5.
\stoplinenote [one]

Then the note will give us, say

3--5 A linenote... 3--5] The lemma is followed by the
actual note.

g. Instead of linenote numeral ranges in the margin, can we bring them into
the textblock? paragraphed notes would need this any way....

h. can we we change the size or style of the linenote numerals? eg, oldstyle or a different font?

II. Bidi issues

1 Bidi: Line numerals enter text. Should mirror the ltr case. I could use a hack

\setuplinenumbering
[width=-1em,
% location=inright,
align=r2l]

the align=r2l is ignored i guess.

but linenote numerals works as expected.

2. \lefttoright and \righttoleft interfere with \startlinenumbering if either
occurs inside the buffer or immediately before --ie no space -- \startbuffer. I wonder why...

3. Line-numeral ranges in linenotes should be reversed in rtl text, eg

110--115
[reading ltr]
=>
115--110
[reading rtl]
