% From example by Pablo to the mailing list. Hyperlinks can have a roll-over but that
% depends on viewer and in the case of acrobat also on a description. Of course it
% add overhead (size) and is not that useful in the end. Ideally we should have the
% option to specify a description but how to interface that ... It's anyway kind of
% trivial to support. Interestingly is that this is currently not widely handled and
% some argue that it then cannot be in a distribution (as some other features only
% supported in commercial viewers).

\setupbodyfont[dejavu]

\setuptagging
  [state=start]

\setupbackend
  [format=pdf/ua-2]

% \nopdfcompression

% \disabledirectives[backends.references.descriptions]           % just link
% \enabledirectives [backends.references.descriptions]           % some detail (some todo)
% \enabledirectives [backends.references.descriptions=reference] % a bit less

\setupinteraction
  [state=start,
   style=,
   color=,
   contrastcolor=,
   display=new]

\protected\def\myurl   #1{\goto{\hyphenatedurl{#1}}[url(#1)]}
\protected\def\myhref#1#2{\goto{\hyphenatedurl{#1}}[url(#2)]}

\starttext

\startTEXpage[pagestate=start, offset=1em, align=center]
    {\bfa Hovering Over Links}
    \blank[1st]
    \attachment[method=hidden,file={signed-a.pdf}]
    \goto{An embedded PDF document}[signed-a::]
    \blank[.5st]
    \myurl{https://contextgarden.net}
    \blank[.5st]
    \myhref{A link to the Garden}{https://contextgarden.net}
\stopTEXpage

\stoptext
