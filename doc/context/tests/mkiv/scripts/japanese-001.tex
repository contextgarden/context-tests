\usemodule[art-01]

% \enabletrackers[scripts.analyzing] % scripts.injections

\definefont[testfont][heiseikakugostd-w5][script=kana,language=jan]
% \definefont[testfont][heiseiminstd-w3]   [script=kana,language=jan]

\starttext

\testfont

\mainlanguage[japanese]
\startscript[nihongo]

% \enabletrackers[scripts.details]
\startlines
             国、『国』、『国』、『国』、『国』、『国』、
\hbox to 3em{国、『国』、『国』、『国』、『国』、『国』、}
             国:国:国:国:国:国:国:国:国:国
\hbox to 3em{国:国:国:国:国:国:国:国:国:国}
\stoplines
% \disabletrackers[scripts.details]

\startbuffer[w3c-fig-3-1]

日本語の表記においては,漢字や仮名だけで
なく,ローマ字やアラビア数字,さらに句読
点や括弧類などの記述記号を用いる.これら
を組み合わせて表す日本語の文章では,表記
上における種々の問題がある.

\stopbuffer

\startbuffer[w3c-fig-3-2]

日本語の表記においては,漢字や仮名だけで
なく,ローマ字やアラビア数字,さらに句読
点や括弧類などの記述記号を用いる。これら
を組み合わせて表す日本語の文章では,表記
上における種々の問題がある。

\stopbuffer

\startbuffer[w3c-fig-3-3]

日本語の表記においては、漢字や仮名だけで
なく、ローマ字やアラビア数字、さらに句読
点や括弧類などの記述記号を用いる。これら
を組み合わせて表す日本語の文章では、表記
上における種々の問題がある。

\stopbuffer

\startbuffer[w3c-fig-3-5]

日本語の表記では“漢字”や”仮名”だけで
なく,“ローマ字”や“アラビア数字”,さら
に“句読点”や“括弧類”などの記述記号を
用いる.

\stopbuffer

\dostepwiserecurse{40}{150}{5} {

    \start

        \page

        \hsize\recurselevel mm

        \getbuffer[w3c-fig-3-1] \blank
        \getbuffer[w3c-fig-3-2] \blank
        \getbuffer[w3c-fig-3-3] \blank
        \getbuffer[w3c-fig-3-5] \blank

        \page

    \stop
}

\stopscript

\stoptext
